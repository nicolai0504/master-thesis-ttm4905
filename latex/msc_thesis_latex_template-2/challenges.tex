\chapter{Challenges}
\label{chp:challenges} 
\section{Large data sets}
When visualizing big data the main challenge is to effectively show the core message of the data. Considering one hour of the data provided from UNINETT, there is almost 400,000 different IP-adresses. And the amounts of flows is in the millions. 

In section \ref{characteristics} good visualization is said to be able to present many numbers in a small space,  make large data sets coherent, and reveal data at several levels of detail. I chose to create individual modules with D3.js, with each covering a different layer of detail. 

\subsection{IP-spectrum}
As mentioned the range of the \gls{ipv4} is large, and with the emergence of \gls{ipv6} there is a challenge present. In \ref{sec:heatmap} this was resolved with pre-processing of the data for a specific task. In other cases such a limitation on the number of IP-addresses represented wouldn't be satisfying.  

\subsection{Increasing number of flows}
The amount of data sent these days are expanding quickly. This means the number of flows will follow, and a visual solution will need to be scalable to handle this increase. In the solution in \ref{sec:d3example}